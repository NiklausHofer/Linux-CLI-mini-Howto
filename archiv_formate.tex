\section{Archiv-Formate}
\subsection{Weshalb braucht es tar}
Archive, komprimiert oder unkomprimiert, gibt es in unz\"ahligen Formaten. Besonders verbreitet unter Linux sind .tar.gz und .tar.bz. Der wesentlich Unterschied von gzip und bzip2 zu zip oder 7z ist, dass sie nur Dateien komprimieren, nicht aber 'zusammenbinden' und in einem Archiv vereinen k\"onnen. Deshalb braucht man dazu tar (typed archive), das resultierende tar archiv wird dann mit gzip oder bzip2 komprimiert. Der tar command bietet aber Parameter und das in einem Prozess zu erledigen.\\
NOTE: .tar.gz ist dasselbe wie .tgz
\subsection{gzip vs. bzip2}
Gzip ist auf absolut jedem Linux bereits vorhanden. Das gilt heute zumeist aus f\"ur bzip2, trotzdem geniesst gzip die h\"ohere Verbreitung. bzip2 ist st\"arker in der Kompression, daf\"ur aber langsamer als gzip.
\subsection{handling archives in the command lines}
Hier sind die meistverwendeten Befehler zum Erstellen und Entpacken der popul\"arsten Archiv Fomrate:
\begin{description}
\item[zip] \hfill \\
\begin{lstlisting}
# zip one or several files
zip new_archive.zip file1 file2
# zip a directory
zip -r new_archive.zip folder1 folder2
# unzip
unzip archive.zip
\end{lstlisting}
\item[rar] \hfill \\
\begin{lstlisting}
# create a rar arive
rar a new_rar_archive.rar folder1 file1
# unrar
unrar x archive.rar
# merge splitted rar archive
unrar x -e file.part1.rar
\end{lstlisting}
\item[7z] \hfill \\
\begin{lstlisting}
# create 7z archive with strongest compression possible
7z a -t7z -m0=lzma -mx=9 -mfb=64 =md=32m -ms=on new_archive.7z folder1
# Dafuer wuerde ich auf jeden Fall einen Alias anlegen
alias 7ze="7z a -t7z -m0=lzma -mx=9 -mfb=64 =md=32m -ms=on"
# Jetzt geht das doch gleich viel einfacher
7ze new_archive.7z folder1
# unzip
7z x file.7z
# gesplittetes Archiv zusammenfuegen
cat *.7z >> output.7z
7z x output.7z
\end{lstlisting}
\item[Tar] \hfill \\
\begin{lstlisting}
# erstellen
tar cfv new_archive.tar folder1 file1 folder2
# entpacken
tar xfv archiv.tar
\end{lstlisting}
\item[gzip] \hfill \\
\begin{lstlisting}
# Datei komprimieren
gzip datei
# dekomprimieren
gunzip datei.gz
# mit tar verpacken und mit gzip verpacken
tar cfvz neues_archiv.tar.gz ordner1 ordner2 datei1
# .tar.gz entpacken
tar xfvz archiv.tar.gz
\end{lstlisting}
\item[bzip2] \hfill \\
\begin{lstlisting}
# Datei komprimieren
bzip2 datei
# Datei dekomprimieren
bunzip2 datei.bz2
# mit tar verpacken und bzip2 komprimieren
tar cfvj neues_archiv.bz2 ordner1 ordner2 datei1
# in einem rutsch entpacken und dekomprimieren
tar xfvj archiv.tar.bz2
\end{lstlisting}

\end{description}
